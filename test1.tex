\documentclass[tate]{jlreq}

\RequirePackage{calc}
\RequirePackage{tikz}
\RequirePackage{KKsymbols}
\usetikzlibrary{shapes}
\usetikzlibrary{calc}

\makeatletter

\begin{document}
\newdimen\width@KKran
\width@KKran=3\zw

\def\AA{.5}
\def\AAA{1}
\def\AAAA{2}\def\AAAB{3}
\def\AAB{4}
\def\AAC{.7pt}
\def\AACA{.3pt}
\def\AACB{.5pt}

\def\AAD{0}
\def\AAE{0}

\NewDocumentCommand{\Daimon@Box@A}{ m m m }{%
  \draw[line width=#1] (0,0) rectangle (#2\width@KKran,- #3\width@KKran);%
  \node[anchor=center,overlay
  % ,rotate=90 % 縦書きなら
  ]
   at (.5*#2\width@KKran,- .5*#3\width@KKran) {\scalebox{1}{\kakko{n}}};
  \pgfmathparse{\AAD + \AA}%
  \edef\AAD{\pgfmathresult}%
}

\NewDocumentCommand{\Shoumon@Box@A}{ m m m }{%
  \draw[line width=#1] (\AAD\width@KKran,-\AAE\width@KKran) -- ++ (#2\width@KKran,0) -- ++ (0,- #3\width@KKran) -- ++(-#2\width@KKran,0);%
  \pgfmathparse{\AAD + #2}%
  \edef\AAD{\pgfmathresult}%
}

\NewDocumentCommand{\Shoumon@Box@B}{ m m m m }{%
  \draw[line width=#1] (\AAD\width@KKran,-\AAE\width@KKran) -- ++ (#3\width@KKran,0);
  \draw[line width=#2, dash pattern=on \AACB off \AACB] (\AAD\width@KKran+#3\width@KKran,-\AAE\width@KKran) -- ++ (0,- #4\width@KKran);
  \draw[line width=#1] (\AAD\width@KKran+#3\width@KKran,-\AAE\width@KKran- #4\width@KKran) -- ++ (-#3\width@KKran,0);
  \pgfmathparse{\AAD + #3}%
  \edef\AAD{\pgfmathresult}%
}

\NewDocumentCommand{\Shoumon@Box@A@P}{ m m m m }{%
  \draw[line width=#1] (\AAD\width@KKran,-\AAE\width@KKran) -- ++ (#3\width@KKran,0);
  \draw[line width=#2, dash pattern=on \AACB off \AACB] (\AAD\width@KKran+#3\width@KKran,-\AAE\width@KKran) -- ++ (0,- #4\width@KKran);
  \draw[line width=#1] (\AAD\width@KKran+#3\width@KKran,-\AAE\width@KKran- #4\width@KKran) -- ++ (-#3\width@KKran,0);
  \node[anchor=center,overlay] at (\AAD\width@KKran+.5*#3\width@KKran,-\AAE\width@KKran- .5*#4\width@KKran) {\scalebox{1}{\kakko{n}}};
  \pgfmathparse{\AAD + #3}%
  \edef\AAD{\pgfmathresult}%
}

\NewDocumentCommand{\GoDown}{}{%
  \pgfmathparse{\AAE + \AAA}%
  \edef\AAE{\pgfmathresult}%
  \def\AAD{0}%
  \pgfmathparse{\AAD + \AA}%
  \edef\AAD{\pgfmathresult}%
}

\begin{tikzpicture}
  \Daimon@Box@A{\AAC}{\AA}{\AAB}%
  \Shoumon@Box@A{\AAC}{\AAAA}{\AAA}\Shoumon@Box@A{\AAC}{\AAAA}{\AAA}\Shoumon@Box@B{\AAC}{\AACA}{\AAAB}{\AAA}\Shoumon@Box@A{\AAC}{\AAAB}{\AAA}\Shoumon@Box@A{\AAC}{\AAAA}{\AAA}\GoDown%
  \Shoumon@Box@A{\AAC}{\AAAA}{\AAA}\Shoumon@Box@A{\AAC}{\AAAB}{\AAA}\Shoumon@Box@A{\AAC}{\AAAA}{\AAA}\Shoumon@Box@A{\AAC}{\AAAA}{\AAA}%
\end{tikzpicture}

\begin{tikzpicture}
  \Daimon@Box@A{\AAC}{\AA}{\AAB}%
  \Shoumon@Box@A@P{\AAC}{\AACA}{\AA}{\AAA}\Shoumon@Box@A{\AAC}{\AAAA}{\AAA}\Shoumon@Box@A{\AAC}{\AAAA}{\AAA}\Shoumon@Box@A{\AAC}{\AAAA}{\AAA}\Shoumon@Box@A{\AAC}{\AAAA}{\AAA}\GoDown%
  \Shoumon@Box@A{\AAC}{\AAAA}{\AAA}\Shoumon@Box@A{\AAC}{\AAAA}{\AAA}\Shoumon@Box@A{\AAC}{\AAAA}{\AAA}\Shoumon@Box@A{\AAC}{\AAAA}{\AAA}%
\end{tikzpicture}
\end{document}


% タスク
% 縦書きの時は問題番号のところを90度回転。
% 上下右のどこを点線にするかをkeyvalで。
% 上下左右どこを太線にするのかをkeyvalで設定できるように。
% 各コマンドの引数はkeyvalに。