\documentclass{jlreq}

\RequirePackage{calc}
\RequirePackage{tikz}
\RequirePackage{KKsymbols}
\usetikzlibrary{shapes}
\usetikzlibrary{calc}

\usepackage{KKran}

\makeatletter

\begin{document}

\begin{tikzpicture}
  \Daimon@Box@A{.5}{6}%
  \Shoumon@Box@A@P{2}\Shoumon@Box@A{3}{2}\Shoumon@Box@A@P{2}\Shoumon@Box@A{5}{2}\Shoumon@Box@A@P{2}\Shoumon@Box@A{3}{2}\GoDown{2}
  \Shoumon@Box@A@P{2}\Shoumon@Box@A{3}{2}\Shoumon@Box@A@P{2}\Shoumon@Box@A{5}{2}\Shoumon@Box@A@P{2}\Shoumon@Box@A{3}{2}\GoDown{2}
  \Shoumon@Box@A@P{1}{1}\Shoumon@Grid@A{5}{40}\Shoumon@Box@A@P{2}\Shoumon@Box@A{3}{2}\Shoumon@Box@A@P{1}{1}\Shoumon@Grid@A{5}{40}
\end{tikzpicture}

\begin{tikzpicture}
  \Daimon@Box@A{.5}{7}%
  \Shoumon@Box@A@P{2}\Shoumon@Box@A{5}{2}\Shoumon@Box@A@P{2}\Shoumon@Box@B{3}{2}\Shoumon@Box@A{3}{2}\GoDown{2}
  \Shoumon@Box@A@P{2}\Shoumon@Box@A{3}{2}\Shoumon@Box@A@P{2}\Shoumon@Box@A{5}{2}\Shoumon@Box@A@P{2}\Shoumon@Box@A{3}{2}\GoDown{2}\Shoumon@Box@A@P{3}\Shoumon@Grid@A{55}{40}
\end{tikzpicture}
\end{document}


% タスク
% 縦書きの時は問題番号のところを90度回転。
% 上下右のどこを点線にするかをkeyvalで。
% 上下左右どこを太線にするのかをkeyvalで設定できるように。
% 各コマンドの引数はkeyvalに。