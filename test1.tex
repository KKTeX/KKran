\documentclass{jlreq}


\usepackage[T1]{fontenc}
\usepackage[kaitouhyouji=1,gridmax=20,kaitoucolor=blue]{KKran}


\begin{document}
\KKran[3pt]{%
  \大問番号[1]{9}
  \小問番号{2}\小問欄{2}{2}[center=\KaitouInput{ア}]\小問番号{2}\小問中欄{5}{2}[west=人物:,center=\KaitouInput{松尾},east=芭蕉]\小問欄{6}{2}[west=作品:,center=\KaitouInput{奥の細道}]\GoDown{2}
  \小問番号{2}\小問欄{4}{2}[center=\KaitouInput{つわもの}]\小問番号{2}\小問欄{2}{2}[center=\KaitouInput{ウ}]\小問番号{2}\小問欄{2}{2}[center=\KaitouInput{エ}]\小問番号{2}\小問欄{4}{2}[center=\KaitouInput{片雲の風}]\小問番号{2}\小問欄{7}{2}[center=\KaitouInput{春夜宴桃李園序}]\GoDown{2}
  \小問番号{4}\小問マス目[45]{65}[\KaitouInput{李白の著名な詩を元にした\ichimoji{「}月日は百代の過客\ichimoji{」}という表現を引いて旅に主題を仄めかしつつ、滑らかにに自身の紀行文に主眼を移す役割。}]\GoDown{4}
  \小問番号{1}\小問マス目{7}[\KaitouInput{道祖神の招き}]
}[shoumon=いろは,daimon=ア]


\MarkArrayMake{テスト}{|$\pm$|0|1|2|3|}
\KKran[3pt]{%
  \大問番号[1]{8}
  \小問番号{1}\小問欄{6}{1}[center=\マークシート{5}{\テストマーク配列}[4]]\GoDown{1}
  \小問番号{1}\小問欄{6}{1}[center=\マークシート{5}{\テストマーク配列}[1]]\GoDown{1}
  \小問番号{1}\小問欄{6}{1}[center=\マークシート{5}{\テストマーク配列}[4]]\GoDown{1}
  \小問番号{1}\小問欄{6}{1}[center=\マークシート{5}{\テストマーク配列}[3]]\GoDown{1}
  \小問番号{1}\小問欄{6}{1}[center=\マークシート{5}{\テストマーク配列}[2]]\GoDown{1}
  \小問番号{1}\小問欄{6}{1}[center=\マークシート{5}{\テストマーク配列}[3]]\GoDown{1}
  \小問番号{1}\小問欄{6}{1}[center=\マークシート{5}{\テストマーク配列}[4]]\GoDown{1}
  \小問番号{1}\小問欄{6}{1}[center=\マークシート{5}{\テストマーク配列}[3]]\GoDown{1}
}[shoumon=いろは,daimon=ア]

\usefont{T1}{phv}{m}{n}$\pi$

\end{document}
