\documentclass[tate]{jlreq}

\usepackage[kaitouhyouji=1,gridmax=20,kaitoucolor=red]{KKran}

\begin{document}
\KKran{%
  \大問番号{9}
  \小問番号{2}\小問欄{2}{2}[center=\KaitouInput{ア}]\小問番号{2}\小問中欄{5}{2}[west=人物:,center=\KaitouInput{松尾},east=芭蕉]\小問欄{6}{2}[west=作品:,center=\KaitouInput{奥の細道}]\GoDown{2}
  \小問番号{2}\小問欄{4}{2}[center=\KaitouInput{つわもの}]\小問番号{2}\小問欄{2}{2}[center=\KaitouInput{ウ}]\小問番号{2}\小問欄{2}{2}[center=\KaitouInput{エ}]
  \小問番号{2}\小問欄{4}{2}[center=\KaitouInput{片雲の風}]
  \小問番号{2}\小問欄{5}{2}[center=\KaitouInput{春夜宴桃李園序}]\GoDown{2}
  \小問番号{4}\小問マス目[45]{65}[\KaitouInput{李白の著名な詩を元にした\ichimoji*{「}月日は百代の過客\ichimoji*{」}という表現を引いて旅に主題を仄めかしつつ\ichimoji*{、}滑らかにに自身の紀行文に主眼を移す役割\ichimoji*{。}}]\GoDown{4}
  \小問番号{1}\小問マス目{7}[\KaitouInput{道祖神の招き}]%
}[shoumon=a,daimon=ア]
\bigskip

\KKran{%
  \大問番号{7}\小問番号{2}\小問欄{3}{2}[center=\KaitouInput{144}]\小問番号{2}\小問欄{4}{2}[south east={$\mathrm{mm}$},center=\KaitouInput{$3.0$}, west={$x=$}]\GoDown{2}
  \小問番号{5}\小問欄{8}{5}[center={\KaitouInput{%
    \begin{tikzpicture}
      \draw[domain=-1.5:1.5, samples=100, thick] plot (\x, {\x*\x});
      \draw[->,, black] (0, -.3) -- (0, 2.5) node[above] {$y$};
      \draw[->,, black] (-2, 0) -- (2, 0) node[right] {$x$};
      \filldraw[black] (0, 0) circle (.5pt) node[below left] {$\mathrm{O}$};
    \end{tikzpicture}}}]
}[daimon=ア]
\bigskip


% 注意
\MarkArrayMake{テストA}{|\MarkFixKK{\zw}{\zw}{$\pm$}|0|1|2|3|}
\MarkArrayMake{テストB}{|ア|イ|ウ|エ|オ|}
\KKran{%
  \大問番号{8}
  \小問番号{1}\小問欄{6}{1}[center=\マークシート{5}{\テストAマーク配列}[4]]\GoDown{1}
  \小問番号{1}\小問欄{6}{1}[center=\マークシート{5}{\テストAマーク配列}[1]]\GoDown{1}
  \小問番号{1}\小問欄{6}{1}[center=\マークシート{5}{\テストAマーク配列}[4]]\GoDown{1}
  \小問番号{1}\小問欄{6}{1}[center=\マークシート{5}{\テストBマーク配列}[3]]\GoDown{1}
  \小問番号{1}\小問欄{6}{1}[center=\マークシート{5}{\テストBマーク配列}[2]]\GoDown{1}
  \小問番号{1}\小問欄{6}{1}[center=\マークシート{5}{\テストBマーク配列}[3]]\GoDown{1}
  \小問番号{1}\小問欄{6}{1}[center=\マークシート{5}{\テストBマーク配列}[4]]\GoDown{1}
  \小問番号{1}\小問欄{6}{1}[center=\マークシート{5}{\テストBマーク配列}[3]]\GoDown{1}
}[shoumon=一,daimon=ア]

\end{document}
