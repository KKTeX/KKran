\documentclass{jlreq}


\usepackage[kaitouhyouji=1,gridmax=25,unit=7mm,kaitoucolor=blue]{KKran}


\begin{document}
\KKran[3pt]{%
  \大問番号[1]{8}
  \小問番号{2}\小問欄{2}{2}[center=\KaitouInput{ア}]\小問番号{2}\小問中欄{5}{2}[west=人物:,center=\KaitouInput{松尾},east=芭蕉]\小問欄{6}{2}[west=作品:,center=\KaitouInput{奥の細道}]\GoDown{2}
  \小問番号{2}\小問欄{4}{2}[center=\KaitouInput{つわもの}]\小問番号{2}\小問欄{2}{2}[center=\KaitouInput{ウ}]\小問番号{2}\小問欄{2}{2}[center=\KaitouInput{エ}]\小問番号{2}\小問欄{4}{2}[center=\KaitouInput{片雲の風}]\小問番号{2}\小問欄{7}{2}[center=\KaitouInput{春夜宴桃李園序}]\GoDown{2}
  \小問番号{3}\小問マス目[45]{65}[\KaitouInput{李白の著名な詩を元にした\ichimoji{「}月日は百代の過客\ichimoji{」}という表現を引いて旅に主題を仄めかしつつ、滑らかにに自身の紀行文に主眼を移す役割。}]\GoDown{3}
  \小問マス目{7}[\KaitouInput{道祖神の招き}]
}[shoumon=いろは,daimon=ア]

\fboxrule=.1pt\fboxsep=0pt
\fbox{\kyouteoval{ば}}\fbox{\kyouteoval{\dccare{jjj}}}\fbox{\kyouteoval{\fbox{\slowcare{j}}}}\fbox{\kyouteoval{\slowcare{a}}}\fbox{\kyouteoval{\fbox{\dccare{Qjj}}}}\fbox{\kyouteoval{ば}}\fbox{\maru{\slowcare{j}}}

\end{document}
