\documentclass[tate]{jlreq}


\usepackage[kaitouhyouji=0]{KKran}

\makeatletter

\begin{document}

\KKran[3pt]{%
  \Daimon@Box@A@N[1]{7}[\kakko{\RevBangou{あ}}]%
  \Shoumon@Box@A@N{2}\Shoumon@Box@A{5}{2}\Shoumon@Box@A@N{2}\Shoumon@Box@B{3}{2}[south east=$\mathrm{mm}$]\Shoumon@Box@A{3}{2}\GoDown{2}
  \Shoumon@Box@A@N{2}\Shoumon@Box@A{3}{2}\Shoumon@Box@A@N{2}\Shoumon@Box@A{5}{2}\Shoumon@Box@A@N{2}\Shoumon@Box@A{3}{2}\GoDown{2}\Shoumon@Box@A@N{3}\Shoumon@Grid@A{48}[47][あ\KaitouInput{ああ\ichimoji{aa}アイエうお\ichimoji{a1}\ichimoji{23}ああ\ichimoji{aa}アイエうおああ}あ\ichimoji{aa}アイエうお\ichimoji{a1}\ichimoji{23}ああ\ichimoji{aa}アイエうお]
}

\SetDaimon{16}
\KKran{%
  \Daimon@Box@A@N{5}%
  \Shoumon@Box@A@N{2}\Shoumon@Box@A{3}{2}\Shoumon@Box@A@N{2}\Shoumon@Box@A{5}{2}\Shoumon@Box@A@N{2}\Shoumon@Box@A{3}{2}\GoDown{2}
  \Shoumon@Box@A@N{2}\Shoumon@Box@A{3}{2}[south east=$\mathrm{mm}$]\Shoumon@Box@A@N{2}\Shoumon@Box@A{5}{2}\Shoumon@Box@A@N[1]{2}\Shoumon@Box@A{3}{2}\GoDown{2}
  \Shoumon@Box@A@N{1}{1}\Shoumon@NonGrid@A{5}\Shoumon@Box@A@N{2}\Shoumon@Box@A{3}{2}\Shoumon@Box@A@N{1}{1}\Shoumon@NonGrid@A{5}[あ\ichimoji{a}あああ]
}[shoumon=A]

\KKran{%
  \Shoumon@Grid@A{55}[42]
}

% \fboxsep=0pt\fboxrule=.1pt
% \noindent%
% \masume@text@KKran{\fbox{あ}\fbox{あ}あああああああああああああああああああああああああああああああああああああああああ\fbox{あ}\fbox{あ}あああああああああああああああああああああああああああああああああああああああああああああああああああああああ}

\end{document}


% タスク
% 縦書きの時は問題番号のところを90度回転。
% 各コマンドの引数はkeyvalに。